\documentclass[17pt,twoside]{extarticle}
\usepackage{lmodern}
\usepackage{amssymb,amsmath}
\usepackage{ifxetex,ifluatex}
\usepackage{fixltx2e} % provides \textsubscript
\ifnum 0\ifxetex 1\fi\ifluatex 1\fi=0 % if pdftex
  \usepackage[T1]{fontenc}
  \usepackage[utf8]{inputenc}
\else % if luatex or xelatex
  \ifxetex
    \usepackage{mathspec}
    \usepackage{xltxtra,xunicode}
  \else
    \usepackage{fontspec}
  \fi
  \defaultfontfeatures{Mapping=tex-text,Scale=MatchLowercase}
  \newcommand{\euro}{€}
\fi
% use upquote if available, for straight quotes in verbatim environments
\IfFileExists{upquote.sty}{\usepackage{upquote}}{}
% use microtype if available
\IfFileExists{microtype.sty}{%
\usepackage{microtype}
\UseMicrotypeSet[protrusion]{basicmath} % disable protrusion for tt fonts
}{}
\ifxetex
  \usepackage[setpagesize=false, % page size defined by xetex
              unicode=false, % unicode breaks when used with xetex
              xetex]{hyperref}
\else
  \usepackage[unicode=true]{hyperref}
\fi
\hypersetup{breaklinks=true,
            bookmarks=true,
            pdfauthor={},
            pdftitle={},
            colorlinks=true,
            citecolor=blue,
            urlcolor=blue,
            linkcolor=magenta,
            pdfborder={0 0 0}}
\urlstyle{same}  % don't use monospace font for urls
\setlength{\parindent}{0pt}
\setlength{\parskip}{6pt plus 2pt minus 1pt}
\setlength{\emergencystretch}{3em}  % prevent overfull lines
\setcounter{secnumdepth}{0}

\usepackage{verses}

\begin{document}

\thispagestyle{empty}

\begin{figure}[htbp]
\centering
\includegraphics[width=15.5cm]{images/frontispiece.png}
\end{figure}

\vspace{4mm}

\begin{enumerate}
\item
  To swim, to dive into the fire, to ride on the curled
  clouds,\footnote{The Tempest 1.2: Ariel. Miraculous action, as quick
    as thought.}\\Whose speechless song being many, seems one.\footnote{Sonnet
    8. Thunder is the speechless song of clouds. \emph{nb. ``Seeming''
    changed to ``seems''.}}
\item
  Time doth transfix the flourish set on youth,\footnote{Sonnet 60.
    \textbf{doth} does \textbf{transfix} pierce \textbf{flourish}
    bloom/adornment}\\Which bounteous gift thou shouldst in bounty
  cherish.\footnote{Sonnet 11. \textbf{thou shouldst} you should
    \textbf{in bounty} generously}
\item
  Be collected; no more amazement. Tell your piteous heart:\footnote{The
    Tempest 1.2: Prospero. \textbf{collected} calm, composed
    \textbf{amazement} fear/wonder}\\Thou art thy mother's glass, and
  she in thee.\footnote{Sonnet 3. \textbf{glass} mirror \textbf{thee}
    you. You are your mother's reflection.}
\item
  So, ere you find where light in darkness lies,\footnote{Love's Labour
    Lost 1.1: Berowne. \textbf{ere} before}\\Gentle breath of yours my
  sails must fill.\footnote{The Tempest, Epilogue: Prospero}
\item
  Grant, if thou wilt, thou art beloved of many,\footnote{Sonnet 10.
    \textbf{grant} allow \textbf{wilt} will \textbf{thou} you
    \textbf{art} are}\\Both in your form and nobleness of
  mind.\footnote{Richard III 3.7: Buckingham}
\item
  Now my charms are all o'erthrown,\footnote{The Tempest Epilogue:
    Prospero. \textbf{charms} spells/enchantments \textbf{o'erthrown}
    been put to an end}\\Begot of nothing but vain fantasy.\footnote{Romeo
    and Juliet 1.4: Mercutio. ``I talk of dreams.'' \textbf{begot}
    brought into existence \textbf{vain} useless}
\item
  Look, whom she best endow'd she gave thee more;\footnote{Sonnet 11.
    \textbf{she} Nature \textbf{best endow'd} gave the best qualities to}\\Our
  fancies are more giddy and unfirm.\footnote{Twelfth Night 2.4: Duke
    Orsino. Here he notes the unsteadiness of man's desires.
    \textbf{fancies} imagined realities \textbf{giddy} disorienting
    \textbf{unfirm} unstable}
\item
  Sap checked with frost and lusty leaves quite gone,\footnote{Sonnet 5.
    Trees in winter. \textbf{checked} congealed \textbf{lusty} healthy
    and strong}\\Courage and hope both teaching him the
  practice.\footnote{Twelfth Night 1.2: Captain. He reassures Viola that
    her brother may have saved himself from drowning.}
\item
  Rough winds do shake the darling buds of May.\footnote{Sonnet 18.
    Inclement weather precedes summer. Adversity at a tender age.}\\I'll
  kneel down, and ask of thee forgiveness.\footnote{King Lear 5.3: King
    Lear. He vows to begin anew with his daughter Cordelia for having
    judged her wrongly. \textbf{thee} you}
\item
  Hourly joys be still upon you!\footnote{The Tempest 4.1: Juno.
    Continue to experience joy in every moment.}\\And frame your mind to
  mirth and merriment.\footnote{The Merchant of Venice 1.2: Messenger.
    \textbf{frame} set \textbf{mirth} cheer/joy}
\item
  The moon new-bent in heaven, shall behold the night\footnote{A
    Midsummer Night's Dream 1.1: Hippolyta. The moon overlooking the
    world at night. \textbf{new-bent moon} crescent moon}\\That has such
  people in't!\footnote{The Tempest 5.1: Miranda. She wonders at
    Alonso's retinue upon his reunion with Ferdinand, after being raised
    by Prospero apart from humanity.}
\item
  And having climb'd the steep-up heavenly hill,\footnote{Sonnet 7. The
    sun rising.}\\Fortune, good night: smile once more: turn thy
  wheel!\footnote{King Lear 2.2: Kent. \textbf{Fortune} goddess of luck
    \textbf{smile} bring me good luck \textbf{thy} your \textbf{wheel}
    Fortune's wheel, which brings luck, misfortune, or neither.}
\item
  Beauty o'ersnow'd and bareness every where\footnote{Sonnet 5. The
    earth at winter. \textbf{o'ersnow'd} covered with snow
    \textbf{bareness everywhere} no leaves on trees, no grass, etc.}\\Thaws
  and resolves itself into a dew.\footnote{Hamlet, Prince of Denmark
    1.2: Hamlet. \textbf{resolves} turns into a different form \emph{nb.
    Added ``s'' to thaw and resolve.}}
\item
  Their eyes do offices of truth, their words are natural breath,
  \footnote{The Tempest 5.1: Prospero. \textbf{do offices of truth}
    reveal what is true}\\All dedicated to closeness and the bettering
  of my mind.\footnote{The Tempest 1.2: Prospero. \textbf{closeness}
    solitude}
\item
  Sound me from my lowest note to the top of my compass:\footnote{Hamlet,
    Prince of Denmark 3.2: Hamlet. \textbf{sound me} test me
    \textbf{lowest note\ldots{}compass} from the bottom of my feet to
    the top of my head}\\My heart is true as steel.\footnote{A Midsummer
    Night's Dream 2.1: Helena.}
\item
  Thy self thy foe, to thy sweet self too cruel;\footnote{Sonnet 1.
    \textbf{thy} your \textbf{foe} enemy}\\I am sure care's an enemy to
  life.\footnote{Twelfth Night 1.3: Sir Toby Belch. \textbf{care} worry}
\item
  Sermons in stones, and good in everything---\footnote{As You Like It
    2.1: Duke Senior. \textbf{sermon} a talk on a religious or moral
    subject}\\And therefore sit you down in gentleness.\footnote{As You
    Like It 2.7: Duke Senior.}
\item
  Have more than thou showest, speak less than thou knowest,\footnote{The
    Tempest 2.1: Antonio. \textbf{thou showest} you show \textbf{thou
    knowest} you know}\\Nor lose possession of that fair thou
  ow'st.\footnote{Sonnet 18. \textbf{fair} beauty \textbf{thou ow'st}
    you own, that is yours}
\item
  What seest thou else in the dark backward and abysm of time?\footnote{The
    Tempest 1.2: Prospero. He asks Miranda to see what she remembers of
    her past. \textbf{seest thou else} else do you see \textbf{abysm}
    bottomless pit}\\So full of shapes is fancy that it alone is high
  fantastical.\footnote{Twelfth Night 1.1: Duke Orsino. \textbf{fancy}
    imagination \textbf{fantastical} remote from reality}
\item
  To give away yourself, keeps yourself still.\footnote{Sonnet 16.}\\Make
  me a willow cabin at your gate.\footnote{Twelfth Night 1.5: Viola. The
    willow cabin represents loyalty.}
\item
  Those be rubies, fairy favours;\footnote{A Midsummer Night's Dream
    2.1: Fairy. He describes the spots on cowslips. \textbf{favours}
    tokens of favour}\\They sparkle still the right Promethean
  fire.\footnote{Love's Labour Lost: 4.3. \textbf{still\ldots{}fire}
    continually with the heavenly fire stolen by Prometheus.}
\item
  Men must endure their going hence, even as their coming
  hither.\footnote{King Lear 5.2: Edgar. No coming, no going. Present
    moment. \textbf{hence} from here \textbf{hither} towards this place.}\\I
  am a fool to weep at what I am glad of.\footnote{The Tempest 2.1:
    Miranda. \textbf{weep} cry}
\item
  Get thee to a nunnery:\footnote{Hamlet, Prince of Denmark 3.1: Hamlet.
    \textbf{thee} you}\\A contract of true love to celebrate.\footnote{The
    Tempest 4.1: Iris. \textbf{contract} a formal agreement to marry}
\item
  If music be the food of love, play on,\footnote{Twelfth Night 1.1:
    Orsino.}\\And let this world no longer be a stage.\footnote{Henry
    the Fourth, Part 2 1.1: Northumberland.}
\item
  A man may see how this world goes with no eyes. Look with thine
  ears:\footnote{King Lear 4.6: King Lear. To the blinded Gloucester.
    \textbf{thine} your}\\The murmuring surge, that on the unnumber'd
  idle pebbles chafes.\footnote{King Lear 4.6: Edgar. To Gloucester.
    \textbf{surge} a sudden powerful forward or upward movement,
    especially by a natural force such as the tide \textbf{chafes} rubs
    abrasively against another}
\item
  And manifest experience had collected\footnote{All's Well That Ends
    Well 1.3: Helena. \textbf{manifest} clear or obvious to the eye or
    mind}\\Of drops that sacred pity hath engend'red.\footnote{As You
    Like It 2.7: Duke Senior. \textbf{engendered} cause or give rise to
    (a feeling, situation, or condition)}
\item
  How beauteous mankind is! O brave new world,\footnote{The Tempest 5.1:
    Miranda. On seeing her betrothed Ferdinand's father Alonso and his
    retinue. \textbf{brave} ready to face and endure danger or pain;
    showing courage}\\Merrily, merrily, shall I live now.\footnote{The
    Tempest 5.1: Ariel. On learning he will soon be freed from his
    service to Prospero. \textbf{merrily} in a cheerful way}
\item
  These our actors, as I foretold you, were all spirits\footnote{The
    Tempest 4.1: Prospero. Explaining his magic arts to Ferdinand.
    \textbf{foretold you} already told you}\\To make me give the lie to
  my true sight.\footnote{Sonnet 150. \textbf{give the lie to} show to
    be inaccurate or untrue}
\item
  With the help of your good hands\footnote{The Tempest 5.1: Prospero.
    Hands that release him from his bonds.} all things in common
  nature\\Should produce without sweat or endeavour.\footnote{The
    Tempest 2.1: Gonzalo. \textbf{common} communal}
\item
  And some donation freely to estate\footnote{The Tempest 4.1: Iris.
    \textbf{donation} gift, favour \textbf{estate} bestow}\\Under the
  blossom that hangs on the bough.\footnote{The Tempest 5.1: Ariel.
    \textbf{bough} main branch of a tree}
\item
  Sounds and sweet airs, that give delight and hurt not:\footnote{The
    Tempest 3.2: Caliban. \textbf{airs} tunes or short melodious songs}\\In
  a cowslip's bell I lie.\footnote{The Tempest 5.1: Ariel.
    \textbf{cowslip} a European primula with clusters of drooping
    fragrant yellow flowers in spring, growing on dry grassy banks and
    in pasture}
\item
  But that the dread of something after death\footnote{Hamlet, Prince of
    Denmark 3.1: Hamlet. \textbf{dread} anticipate with great
    apprehension or fear}\\It droppeth as the gentle rain from
  heaven.\footnote{The Merchant of Venice 4.1: Portia. She speaks of the
    quality of mercy. \textbf{droppeth} drops}
\item
  And ye that on the sands with printless foot\\Do chase the ebbing
  Neptune---\footnote{Richard III 5.5: Henry, Earl of Richmond.
    \textbf{ye} you all \textbf{printless foot} footsteps that leave no
    mark \textbf{ebbing} receding, moving away}\\O, let me see thee
  walk! Thou dost not halt.\footnote{Henry IV, Part 1 2.4: Falstaff.
    \textbf{thee} you \textbf{thou} you \textbf{halt} stop}
\item
  But, like a cloistress, she will veiled walk\footnote{Twelfth Night
    1.1: Valentine. \textbf{cloistress} nun \textbf{veil} a piece of
    fabric forming part of a nun's headdress, resting on the head and
    shoulders}\\Out of the jaws of death.\footnote{Twelfth Night 3.4:
    Antonio.}
\item
  As there is sense in truth and truth in virtue,\footnote{Measure For
    Measure 5.1: Mariana.}\\Joy, gentle friends, joy and fresh days of
  love accompany your hearts!\footnote{A Midsummer Night's Dream 5.1:
    Theseus. \textbf{fresh} new}
\item
  For who would bear the whips and scorns of time,\footnote{Hamlet,
    Prince of Denmark 3.1: Hamlet. \textbf{scorns} feelings and
    expressions of contempt or disdain for someone or something}\\And
  kiss the lips of unacquainted change?\footnote{King John 3.4:
    Pandulph. \textbf{unacquainted} not having met before}
\item
  Often have you heard that told:\footnote{The Merchant of Venice 2.7:
    Morocco.} Wherefore are these things hid?\\Wherefore have these
  gifts a curtain before 'em?\footnote{Twelfth Night 1.3: Sir Toby
    Belch. \textbf{wherefore} for what reason}
\item
  And summer's lease hath all too short a date:\footnote{Sonnet 18.
    \textbf{lease} a contract by which one party conveys land, property,
    services, etc. to another for a specified time \textbf{hath} has
    \textbf{all too short a date} will end soon.}\\The hour's now come;
  the very minute bids thee ope thine ear.\footnote{The Tempest 1.2:
    Prospero. He reveals to Miranda her past. \textbf{bids} asks
    \textbf{thee} you \textbf{ope} open \textbf{thine ear} your ear}
\item
  I am all the daughters of my father's house, and all the brothers
  too.\footnote{Twelfth Night 2.4: Viola.}\\O spirit of love! how quick
  and fresh art thou.\footnote{Twelfth Night 1.1: Duke Orsino.
    \textbf{quick} alive \textbf{art thou} you are}
\item
  And enterprises of great pith and moment\footnote{Hamlet, Prince of
    Denmark 3.1: Hamlet. \textbf{enterprises} projects \textbf{pith}
    substance, essence \textbf{moment} urgency}\\Are melted into air,
  into thin air.\footnote{The Tempest 4.1: Prospero. What becomes of his
    conjured spirits.}
\item
  O, swear not by the moon, the inconstant moon,\footnote{Romeo and
    Juliet 2.2: Juliet. Her response to Romeo's avowals. \textbf{swear}
    vow \textbf{inconstant} frequently changing; variable or irregular}\\If
  it be not now, yet it will come: the readiness is all.\footnote{Hamlet,
    Prince of Denmark 5.2: Hamlet. \textbf{readiness} the state of being
    fully prepared for something}
\item
  And fearless minds climb soonest unto crowns\footnote{Henry VI, Part
    III 4.7: Gloucester. \textbf{crowns} the top or highest part of
    something, i.e.~awakening}\\That show, contain and nourish all the
  world.\footnote{Love's Labour Lost 4.3: Biron.}
\item
  Youth's a stuff will not endure;\footnote{Twelfth Night 2.3: Feste.
    Youth is impermanent and subject to aging.}\\O, out of that `no
  hope' what great hope have you!\footnote{The Tempest 2.1: Antonio.}
\item
  And thus the native hue of resolution\footnote{Hamlet, Prince of
    Denmark 3.1: Hamlet. \textbf{native hue of resolution} original
    aspiration}\\Lies rich in virtue and unmingled.\footnote{Troilus and
    Cressida 1.3: Agamemnon. \textbf{unmingled} not mixed, pure}
\item
  Happiness courts thee in her best array\footnote{Romeo and Juliet 3.3:
    Friar John. \textbf{courts thee} woos thee, seeks your favour
    \textbf{array} elaborate or beautiful clothing}\\And joy comes well
  in such a needy time.\footnote{Romeo and Juliet 3.5: Juliet.
    \textbf{comes well} is welcome \textbf{needy} insecure}
\item
  Nature's bequest gives nothing but doth lend;\footnote{Sonnet 4.
    \textbf{bequest} legacy \textbf{gives\ldots{}lend} is merely
    borrowed}\\Thy truth, then, be thy dower.\footnote{King Lear 1.1:
    King Lear. Despite Cordelia's honesty, Lear does not perceive her
    faithfulness to him. These verses incite us to engage with truth as
    a test of faith, leaving behind the dower of possessions.
    \textbf{thy truth} your honesty \textbf{dower} life estate to which
    a woman is entitled on the death of her husband}
\item
  Study is like the heaven's glorious sun,\footnote{Love's Labour Lost
    1.1: Berowne. True study brings clarity. \textbf{glorious} having a
    striking beauty or splendour}\\Which touch'd the very virtue of
  compassion in thee.\footnote{The Tempest 1.2: Prospero. \textbf{thee}
    you}
\item
  Uttering such dulcet and harmonious breath\footnote{A Midsummer
    Night's Dream 2.1: Oberon. \textbf{uttering} speaking
    \textbf{dulcet} sweet and soothing}\\That long have frown'd upon
  their enmity!\footnote{Richard III 5.5: Richmond. Brotherhood and
    peace to go beyond strife. \textbf{that long} that for a long time
    \textbf{have frowned} have disapproved of \textbf{enmity} a state or
    feeling of active opposition or hostility}
\item
  Give me your hands, if we be friends;\footnote{A Midsummer Night's
    Dream 5.1: Puck.}\\We are such stuff as dreams are made
  on.\footnote{The Tempest 4.1: Prospero. On the insubstantiality of
    phenomenal objects.}
\item
  And nothing 'gainst Time's scythe can make defence---\footnote{Sonnet
    12. Impermanence. \textbf{scythe} a tool used for cutting crops such
    as grass or corn, with a long curved blade at the end of a long pole
    attached to one or two short handles.}\\Herein lives wisdom, beauty
  and increase.\footnote{Sonnet 11. Touching impermanence we get wisdom,
    and our love increases. \textbf{herein} in this insight}
\item
  I must go seek some dewdrops here;\footnote{A Midsummer Night's Dream
    2.1: Fairy.}\\It blesseth him that gives and him that
  takes.\footnote{The Merchant of Venice 4.1: Portia. On mercy
    (compassion).}
\item
  I put you to the use of your own virtues.\footnote{All's Well That
    Ends Well 5.1: Helena.}\\All things are ready, if our minds be
  so.\footnote{Henry V 4.3: King Henry}
\item
  Now stand you on the top of happy hours,\footnote{Sonnet 16.}\\Against
  the stormy gusts of winter's day.\footnote{Sonnet 13. \textbf{gust} a
    sudden strong rush of wind}
\item
  Let gentleness my strong enforcement be\footnote{As You Like It 2.7:
    Orlando}\\To take a new acquaintance of thy mind.\footnote{Sonnet
    77.}
\item
  To take arms against a sea of troubles, and by opposing end
  them?\footnote{Hamlet, Prince of Denmark 3.1: Hamlet.}\\Let it not
  enter in your mind of love.\footnote{Merchant of Venice 2.8: Salerio.}
\item
  These most brisk and giddy-paced times:\footnote{Twelfth Night 2.4:
    Duke Orsino. \textbf{brisk} active and energetic
    \textbf{giddy-paced} disorienting and alarming times}\\Is man no
  more than this? Consider him well.\footnote{King Lear 3.4: King Lear.}
\item
  Who with thy saffron wings upon my flowers\footnote{The Tempest 4.1:
    Ceres. \textbf{saffron} an orange-yellow flavouring, food colouring,
    and dye made from the dried stigmas of a crocus.}\\Calls back the
  lovely April of her prime:\footnote{Sonnet 3. \textbf{prime} the state
    or time of greatest vigour or success in a person's life} the form
  of my intent.\footnote{Twelfth Night 1.2: Viola. Beginner's mind,
    aspiration.}
\item
  It is a wise father that knows his own child,\footnote{The Merchant of
    Venice 2.2: Launcelot.}\\Like to a double cherry, seeming parted,
  but yet an union in partition.\footnote{A Midsummer Night's Dream 3.2:
    Helena.}
\item
  In action how like an angel! In apprehension how like a god,
  \footnote{Hamlet, Prince of Denmark 2.2: Hamlet. He speaks of man.
    \textbf{apprehension} understanding}\\That the rude sea grew civil
  at her song.\footnote{A Midsummer Night's Dream 2.1: Oberon. Of a
    mermaid on a dolphin's back. \textbf{rude} rough, choppy
    \textbf{civil} calm}
\item
  Gaze where you should, and that will clear your sight.\footnote{Comedy
    of Errors 3.2: Luciana. \textbf{gaze} look steadily and intently}\\Enrich
  the time to come with smooth-fac'd peace.\footnote{Richard III 5.5:
    Henry, Earl of Richmond.}
\item
  The slings and arrows of outrageous fortune---\footnote{Hamlet, Prince
    of Denmark 3.1: Hamlet. Suffering resulting from past actions.
    \textbf{slings} a simple weapon in the form of a strap or loop, used
    to hurl stones \textbf{outrageous} very bold and unusual and rather
    shocking}\\These blessed candles of the night.\footnote{The Merchant
    of Venice 5.1: Bassanio. The stars.}
\item
  O, from what power hast thou this powerful might,\footnote{Sonnet 150.}\\By
  chance or nature's changing course untrimm'd?\footnote{Sonnet 18. The
    insight of impermanence gives us power over our lives.
    \textbf{untrimmed} not having been adjusted (a sail) to take
    advantage of the wind}
\item
  Rise from the ground like feathered Mercury,\footnote{Henry IV, Part 1
    4.1: Vernon.}\\Then to the elements be free, and fare thou
  well!\footnote{The Tempest, 5.1: Prospero. \textbf{fare thou well} go
    well, have a good trip}
\item
  The constancy and virtue of your love---\footnote{Sonnet 117}\\Diffusest
  honey-drops, refreshing showers.\footnote{The Tempest 4.1: Ceres.
    \textbf{diffusest} spread over a wide area}
\item
  For never-resting time leads summer on---\footnote{Sonnet 5. Time here
    is impermanence.}\\The wheel is come full circle: I am
  here.\footnote{King Lear 5.3: Edmund. On discovering his half-brother
    Edgar.}
\item
  But how is it that this lives in thy mind,\footnote{The Tempest 1.2:
    Prospero}\\The undiscover'd country from whose bourn no traveller
  returns? \footnote{Hamlet, Prince of Denmark 3.1: Hamlet. What is
    beyond death. \textbf{bourn} boundary}
\item
  They are the books, the arts, the academes---\footnote{Love's Labour
    Lost: 4.3. \textbf{academes} academies. They put their life into
    books, arts, or intellectual knowledge.}\\And I serve the fairy
  queen.\footnote{A Midsummer Night's Dream 2.1: Fairy.}
\item
  Smooth runs the water where the brook is deep.\footnote{The Tempest,
    Epilogue: Prospero.}\\What stronger breastplate than a heart
  untainted?\footnote{The Tempest 2.1: Gonzalo. \textbf{breastplate} a
    piece of armour covering the chest \textbf{untainted} not
    contaminated or polluted}
\item
  Then wisely, good sir, weigh our sorrow with our comfort,\footnote{The
    Tempest 2.1: Gonzalo.}\\That ebb and flow by the moon.\footnote{King
    Lear 5.3: King Lear.}
\item
  All that glisters is not gold. To plainness honour's bound\footnote{The
    Merchant of Venice 2.7. \textbf{glisters} sparkles, glitters
    \textbf{bound} restricted or confined to}\\When majesty falls to
  folly.\footnote{King Lear 1.1: Kent. King Lear is caught in the wrong
    view that his daughter Cordelia is not grateful to him. Kent,
    knowing her faithfulness, tries to intervene. \textbf{bound}
    restricted or confined to \textbf{majesty} impressive beauty, royal
    power \textbf{folly} lack of good sense, foolishness}
\item
  Think'st thou I'd make a life of jealousy?\footnote{Othello 3.3:
    Othello. Iago plants false seeds in Othello of his wife's
    unfaithfulness. Othello says he will not live in jealousy.}\\The
  quality of mercy is not strain'd.\footnote{The Merchant of Venice 4.1:
    Portia. Compassion frees us from the bonds of jealousy, and it is
    not difficult at all. \textbf{not strained} not artificial or forced}
\item
  O heaven, O earth, bear witness to this sound,\footnote{The Tempest
    2.1: Miranda. \textbf{bear witness to} testify to}\\As full of
  spirit as the month of May.\footnote{Henry IV, Part 1 4.1: Vernon.}
\item
  When I consider every thing that grows\\Holds in perfection but a
  little moment---\footnote{Sonnet 15.}\\Pray you, tread softly, that
  the blind mole may not hear a foot fall.\footnote{The Tempest 4.1:
    Caliban. \textbf{mole} a small burrowing mammal with dark velvety
    fur, a long muzzle, and very small eyes, feeding mainly on worms,
    grubs, and other invertebrates. \textbf{foot fall} footstep}
\item
  With gentle conference, soft and affable,\footnote{Taming of the Shrew
    2.1: Petruchio. \textbf{conference} conversation, speech
    \textbf{affable} friendly, good-natured, or easy to talk to}\\Let
  your indulgence set me free.\footnote{The Tempest, Epilogue: Prospero.
    \textbf{indulgence} the state or attitude of being indulgent or
    tolerant}
\item
  Light, seeking light, doth light of light beguile;\footnote{Love's
    Labour Lost 1.1: Berowne. \textbf{Light} i.e., eyes \textbf{light}
    enlightenment \textbf{light\ldots{}beguile} we are cheated out of
    enlightenment by excessive searching}\\Now let not Nature's hand
  keep the wild flood confin'd!\footnote{Henry IV, Part 2 1.1:
    Northumberland.}
\item
  True, I talk of dreams,\footnote{Romeo and Juliet 1.4: Mercutio. This
    follows Romeo's interruption on his depiction of Queen Mab, who
    tempts men and women with desires in their sleep.} for there is
  nothing\\Either good or bad, but thinking makes it so.\footnote{Hamlet,
    Prince of Denmark 2.2: Hamlet. In conversation with Guildenstern he
    sees Denmark as a prison, but recognizes that this it the product of
    his own thinking.}
\item
  What's in a name? that which we call a rose\footnote{Romeo and Juliet
    2.2: Juliet. She sees the illusory nature of the world of name and
    form.}\\Being once display'd doth fall that very hour.\footnote{Twelfth
    Night 2.4: Orsino.}
\item
  O, if you but knew how you the purpose cherish!\footnote{The Tempest,
    Epilogue: Prospero. \textbf{the purpose cherish} hold to your deep
    aspiration}\\If all were minded so, the times should
  cease.\footnote{Sonnet 11.}
\item
  What is love? 'tis not hereafter,\footnote{Twelfth Night 2.3: Feste.
    \textbf{hereafter} at some time in the future}\\And being frank she
  lends to those are free.\footnote{Sonnet 4. \textbf{frank} open,
    sincere, or undisguised}
\item
  What's to come is still unsure;\footnote{Twelfth Night 2.3: Feste.}
  what's past is prologue.\footnote{The Tempest 2.1: Antonio.
    \textbf{prologue} an event or act that leads to another}\\Present
  mirth hath present laughter.\footnote{Twelfth Night 2.3: Feste.
    \textbf{mirth} cheerfulness, joyfulness}
\item
  And the moon changes even as your mind,\footnote{The Taming of the
    Shrew 4.5: Katherina.}\\But thy eternal summer shall not
  fade.\footnote{Sonnet 18. \textbf{eternal} undying, immortal}
\item
  I, thus neglecting worldly ends,\footnote{The Tempest 1.2: Prospero.
    \textbf{neglecting} not paying attention to}\\Play out the
  play.\footnote{Henry IV, Part 1 2.4: Falstaff.}
\item
  Continue still in this so good a mind,\footnote{Henry VI, Part II 4.9:
    King Henry.}\\Wherein it finds a joy above the rest.\footnote{Sonnet
    91.}
\item
  To forswear the full stream of the world\\And to live in a nook merely
  monastic\footnote{As You Like It 3.2: Rosalind. \textbf{nook} a corner
    or recess, especially one offering seclusion or security}\\And by my
  body's action teach my mind.\footnote{Coriolanus 3.2: Coriolanus.}
\item
  Defer no time, delays have dangerous ends;\footnote{Henry VI, Part 1:
    Alençon. \textbf{defer} put off (an action or event) to a later
    time; postpone \textbf{ends} a termination of a state or situation}\\Thou
  shalt be as free as mountain winds.\footnote{The Tempest 1.2:
    Prospero.}
\item
  Understanding begins to swell \footnote{The Tempest 5.1: Prospero.
    \textbf{swell} become or make greater in intensity, number, amount,
    or volume} by prayer, which pierces so\\That it assaults mercy
  itself, and frees all faults.\footnote{The Tempest, Epilogue:
    Prospero. \textbf{assaults} make a physical attack on \textbf{mercy}
    compassion}
\item
  As it is a spare life, look you, it fits my humour well,\footnote{As
    You Like It 3.2: Touchstone. \textbf{spare} frugal \textbf{humour}
    temperament}\\With smiling plenty, and fair prosperous
  days!\footnote{Richard III 5.5: Richmond.}
\item
  Th'endeavour of this present breath may buy\footnote{Love's Labour
    Lost 1.1: King. \textbf{endeavour} attempt to achieve a goal}\\The
  very lifeblood of our enterprise.\footnote{Henry IV, Part 1 4.1:
    Hotspur. \textbf{lifeblood} the indispensable factor or influence
    that gives something its strength and vitality}
\item
  But I will tarry; the fool will stay, and let the wise man
  fly,\footnote{King Lear 2.4: Fool. \textbf{tarry} stay longer than
    intended; delay leaving a place \textbf{fly} depart hastily}\\To pay
  this debt of love but to a brother.\footnote{Twelfth Night 1.1:
    Orsino.}
\item
  And now let's go hand in hand, not one before another,\footnote{Comedy
    of Errors 5.1: Dromio of Ephesus.}\\Swifter than the moon's
  sphere.\footnote{A Midsummer Night's Dream 2.1: Fairy.}
\item
  Smiling at grief:\footnote{Twelfth Night 2.4: Viola.} Awake,
  awake!\footnote{The Tempest 2.1: Ariel.}\\In delay there lies no
  plenty.\footnote{Twelfth Night 2.3: Feste. \textbf{plenty} profit}
\item
  And then the moon, like to a silver bow\footnote{A Midsummer Night's
    Dream 1.1: Hippolyta. The moon overlooking the world at night.}\\Upon
  the place beneath: it is twice blest.\footnote{The Merchant of Venice
    4.1: Portia. The light of the moon is the light of compassion,
    lighting the moon and the earth below.}
\item
  And as the morning steals upon the night,\footnote{The Tempest 5.1:
    Prospero. \textbf{steals} move somewhere quietly or surreptitiously}\\Consideration
  like an angel came.\footnote{The Tempest 2.1: Antonio.
    \textbf{consideration} mindfulness and sensitivity towards others}
\item
  When we have shuffled off this mortal coil,\footnote{Hamlet, Prince of
    Denmark 3.1: Hamlet. \textbf{shuffled off this mortal coil} become
    free of ideas about birth and death}\\There's nothing ill can dwell
  in such a temple.\footnote{The Tempest 1.2: Miranda.}
\item
  Be not afeard; the isle is full of noises,\footnote{The Tempest 3.2:
    Caliban. \textbf{afeard} afraid}\\To entrap the wisest.\footnote{The
    Merchant of Venice 3.2: Bassanio.}
\item
  Roses have thorns, and silver fountains mud---\footnote{Sonnet 35}\\I
  would you would make use of that good wisdom.\footnote{King Lear 1.4:
    Goneril.}
\item
  Make the babbling gossip of the air cry out:\footnote{Twelfth Night
    1.5: Viola. \textbf{babbling} (of a stream) make the continuous
    murmuring sound of water flowing over stones \textbf{gossip} casual
    or unconstrained conversation or reports about other people}\\`There
  are occasions and causes, why and wherefore in all things!' \footnote{Henry
    V 5.1: Fluellen. \textbf{wherefore} for what reason \emph{nb. ``is''
    changed to ``are''}}
\item
  Or to thyself at least kind-hearted prove:\footnote{Sonnet 10.}\\As
  fast as thou shalt wane, so fast thou growest.\footnote{Sonnet 11.
    \textbf{wane} (of the moon) have a progressively smaller part of its
    visible surface illuminated, so that it appears to decrease in size.}
\item
  For virtue and true beauty of the soul,\footnote{Henry VIII 4.2:
    Katherine.}\\Halloo your name to the reverberate hills!\footnote{Twelfth
    Night 1.5: Viola. \textbf{Halloo} cry or shout `halloo' to attract
    attention \textbf{reverberate} (of a place) appear to vibrate
    because of a loud noise}
\item
  But we in silence hold this virtue well:\footnote{Troilus and Cressida
    4.1: Paris.}\\The amity that wisdom knits not, folly may easily
  untie.\footnote{Troilus and Cressida 2.3: Ulysses. \textbf{amity}
    friendly relations \textbf{knits} causes to unite \textbf{folly}
    lack of good sense, foolishness}
\item
  Thy virtues spoke of, and thy beauty sounded,\footnote{Taming of the
    Shrew 2.1: Petruchio. \textbf{sounded} tested}\\The better part of
  valour is discretion.\footnote{Henry IV, Part I 5.4: Falstaff.
    \textbf{valour} great courage in the face of danger
    \textbf{discretion} the freedom to decide what should be done in a
    particular situation}
\item
  Draw the curtain close and let us all to meditation,\footnote{Henry
    VI, Part II 3.3: King Henry.}\\To pluck bright honour from the
  pale-fac'd moon.\footnote{Henry IV, Part I 1.3: Hotspur.
    \textbf{pluck} take hold of (something) and quickly remove it from
    its place}
\item
  My crown is call'd content. A crown it is that seldom kings
  enjoy.\footnote{Henry VI, Part III 3.1: King Henry.}\\Silence bestows
  that virtue on it.\footnote{Merchant of Venice 5.1: Nerissa.
    \textbf{bestows} confers or presents (an honour, right, or gift)}
\item
  Time travels in divers paces with divers persons,\footnote{As You Like
    It 3.2: Rosalind. \textbf{diverse} different \textbf{paces} speeds}\\And,
  since I saw thee, th' affliction of my mind amends.\footnote{The
    Tempest 5.1: Alonso. \textbf{amends} is put right}
\item
  When wheat is green, when hawthorn buds appear,\footnote{A Midsummer
    Night's Dream 1.1: Helena.}\\These vacant leaves thy mind's imprint
  will bear.\footnote{Sonnet 77. \textbf{vacant leaves} empty pages
    \textbf{mind's imprint} writing}
\item
  Burd'ned with like weight of pain,\footnote{Comedy of Errors 2.1:
    Adriana. \textbf{burdened} loaded heavily}\\Thou didst smile,
  infused with a fortitude from heaven.\footnote{The Tempest 1.2:
    Prospero. \textbf{infused} filled, pervaded \textbf{fortitude}
    courage in pain or adversity}
\item
  This bud of love, by summer's ripening breath---\footnote{Romeo and
    Juliet 2.2: Juliet.}\\Was it not to refresh the mind of
  man?\footnote{The Taming of the Shrew 3.1: Lucentio.}
\item
  So shaken as we are, so wan with care---\footnote{Henry IV, Part I
    1.1: King Henry. \textbf{wan} (of a person's complexion or
    appearance) pale and giving the impression of illness or exhaustion}\\Awake,
  dear heart, awake; thou hast slept well. Awake!\footnote{The Tempest
    1.2: Prospero.}
\item
  Enforce attention like deep harmony;\footnote{Richard II 2.1: Gaunt.
    \textbf{enforce} cause (something) to happen by necessity}\\You
  shall find your safety manifested.\footnote{Measure For Measure 4.3:
    Duke. \textbf{manifested} clear or obvious to the eye or mind}
\item
  Hath not in nature's mystery more science\footnote{All's Well That
    Ends Well 5.3: King.}\\To make the coming hour o'erflow with
  joy?\footnote{All's Well That Ends Well 2.4: Parolles.
    \textbf{o'erflow} overflow}
\item
  How hard it is to hide the sparks of nature!\footnote{Cymbeline 3.3:
    Belarius. \textbf{sparks of nature} signs of inherent qualities}\\Virtue
  is bold, and goodness never fearful.\footnote{Measure For Measure 3.1:
    Duke. \textbf{bold} showing a willingness to take risks; confident
    and courageous}
\item
  I will believe thou hast a mind that suits\footnote{Twelfth Night 1.2:
    Viola. \textbf{suits} be convenient for or acceptable to}\\And may
  enjoy such quiet walks as these.\footnote{Henry VI, Part II 4.10:
    Iden.}
\item
  Who doth ambition shun, and loves to live i' th' sun,\footnote{As You
    Like It 2.5: Song. \textbf{ambition} desire and determination to
    achieve success \textbf{shun} persistently avoid, ignore, or reject
    (someone or something) through antipathy or caution \textbf{i' th'
    sun} in the sun}\\He finds the joys of heaven here on
  earth.\footnote{Merchant of Venice 3.5: Jessica.}
\item
  Enjoy thy plainness. It nothing ill becomes thee.\footnote{Antony and
    Cleopatra 2.4: Pompey.}\\For 'tis the mind that makes the body
  rich.\footnote{The Taming of the Shrew 4.3: Petruchio.}
\item
  Crowning the present, doubting of the rest?\footnote{Sonnet 115.
    \textbf{crowning the present} resting in the present moment}\\Keep
  unshaked that temple, thy fair mind.\footnote{Cymbeline 2.1: Second
    Lord.}
\item
  Unlooked for joy in that I honour most:\footnote{Sonnet 25.
    \textbf{Unlooked for} unexpectedly/overlooked, disregarded
    \textbf{joy\ldots{}most} enjoy what I consider most worthy of honour}\\Your
  bounty, virtue, fair humility.\footnote{Richard III 3.7: Buckingham.
    \textbf{bounty} something given or occurring in generous amounts}
\item
  Divert strong minds to the course of alt'ring things.\footnote{Sonnet
    115. \textbf{divert} cause (someone or something) to change course
    or turn from one direction to another \textbf{altering things}
    things that change in character or composition, typically in a
    comparatively small but significant way}\\Where words are scarce,
  they are seldom spent in vain.\footnote{Richard II 2.1: Gaunt.
    \textbf{scarce} occurring in small numbers or quantities; rare
    \textbf{seldom} not often; rarely \textbf{vain} having no likelihood
    of fulfilment}
\item
  For virtue's office never breaks men's troth,\footnote{Love's Labour
    Lost 5.2: Princess. \textbf{office} function \textbf{troth} faith or
    loyalty pledged}\\Nor hath Love's mind of any judgment
  taste.\footnote{A Midsummer Night's Dream 1.1: Helena.}
\item
  As Nature was in making graces dear,\footnote{Love's Labour Lost 2.1:
    Boyet. \textbf{dear} costly}\\Then happy I that love and am
  beloved.\footnote{Sonnet 25.}
\item
  You bear a gentle mind, and heav'nly blessings follow such
  creatures.\footnote{Henry VIII 2.3: Chamberlain.}\\Steel thy fearful
  thoughts and change misdoubt to resolution. \footnote{Henry VI, Part
    II 3.1: York. \textbf{steel} mentally prepare (oneself) to do or
    face something difficult \textbf{misdoubt} mistrust, uncertainty}
\item
  Do not infest your mind with beating on\\The strangeness of this
  business;\footnote{The Tempest 5.1: Prospero. \textbf{infest}
    overwhelm \textbf{beating} dwelling}\\It is the purpose that makes
  strong the vow.\footnote{Troilus and Cressida 5.3: Cassandra.
    \textbf{purpose} intention, volition}
\item
  A turn or two I'll walk to still my beating mind.\footnote{The Tempest
    4.1: Prospero. \textbf{A turn or two I'll walk} I'll walk around a
    bit \textbf{beating mind} mind overwhelmed with emotion}\\My crown
  is in my heart, not on my head.\footnote{Henry VI, Part III 3.1: King
    Henry.}
\item
  That love which virtue begs and virtue grants\footnote{Henry VI, Part
    III 3.2: Lady Grey. \textbf{begs} asks for}\\Is true of mind and
  made of no such baseness.\footnote{Othello 3.4: Desdemona.
    \textbf{baseness} lack of moral principles}
\item
  Your patience and your virtue well deserves it\footnote{As You Like It
    5.4: Jaques.}\\That every eye which in this forest looks\\Shall see
  thy virtue witness'd every where.\footnote{As You Like It 3.2:
    Orlando. \textbf{witnessed} openly shown}
\item
  Cease, cease these jars and rest your minds in peace\footnote{Henry
    VI, Part I 1.1: Bedford. \textbf{jars} discord or disagreements}\\And
  take thou my oblation, poor but free.\footnote{Sonnet 125.
    \textbf{oblation} a thing presented or offered to God or a god}
\item
  To make you understand this in a manifested effect:\footnote{Measure
    For Measure 4.2: Duke. \textbf{manifested} clear or obvious to the
    eye or mind}\\Now you are heir, therefore enjoy it now.\footnote{Henry
    VI, Part III 1.2: Edward. \textbf{heir} a person who inherits and
    continues the work of a predecessor}
\item
  The purest spring is not so free from mud;\footnote{Henry VI, Part II
    3.1: Gloucester.}\\It is the show and seal of nature's
  truth.\footnote{All's Well That Ends Well 1.3: Countess. \textbf{seal}
    a thing regarded as a confirmation or guarantee of something}
\item
  Comets, importing change of times and states---\footnote{Henry VI,
    Part I 1.1: Bedford. \textbf{importing} indicating or signifying}\\O
  infinite virtue, com'st thou smiling from\\The world's great snare
  uncaught?\footnote{Antony and Cleopartra 4.8: Cleopatra.
    \textbf{snare} a trap for catching birds or mammals; a thing likely
    to lure or tempt someone into harm or error}
\item
  The very virtue of compassion in thee\footnote{The Tempest 1.2:
    Prospero.}\\Shall change all griefs and quarrels into
  love.\footnote{Henry V 5.2: Queen Isabella. \textbf{quarrel} an angry
    argument or disagreement}
\item
  You see how all conditions, how all minds tender down their
  services?\footnote{Timon of Athens 1.1: Poet. \textbf{tender down}
    offer \emph{nb. interceding clause removed ``You see how all
    conditions, how all minds, / As well of glib and slipp'ry creatures
    as / Of grave and austere quality, tender down / Their services to
    Lord Timon.''}}\\Silence is the perfectest herald of joy.\footnote{Much
    Ado About Nothing 2.1: Claudio. \textbf{herald} a person or thing
    viewed as a sign that something is about to happen}
\item
  All of one nature, of one substance bred,\footnote{Henry IV, Part I
    1.1: King. \textbf{bred} reared or raised in a specified environment
    or way}\\When inward joy enforc'd my heart to smile!\footnote{Two
    Gentlemen of Verona 1.2: Julia. \textbf{enforced} caused by
    necessity}
\item
  Who alone suffers suffers most i' th' mind,\footnote{King Lear 3.6:
    Edgar. \textbf{i' th' mind} in the mind}\\Then music with her silver
  sound\\With speedy help doth lend redress.\footnote{Romeo and Juliet
    4.5: Peter. \textbf{doth lend redress} does offer remedy or
    compensation for a wrong or grievance}
\item
  An odorous chaplet of sweet summer buds\footnote{A Midsummer Night's
    Dream 2.1: Titania. \textbf{odorous} having or giving off a
    fragrance \textbf{chaplet} a garland or circlet for a person's head}\\Whereof
  the root was fix'd in virtue's ground.\footnote{Henry VI, Part III
    3.3: Warwick.}
\item
  One feast, one house, one mutual happiness,\footnote{Two Gentlemen of
    Verona 5.4: Valentine.}\\With profits of the mind, study and
  fast.\footnote{Measure for Measure 1.4: Lucio. \textbf{profits}
    advantages; benefits \textbf{fast} abstain from all or some kinds of
    food or drink}
\item
  The griefs are ended by seeing the worst,\footnote{Othello 1.3: Duke.}\\Then
  sigh not so, but let them go.\footnote{Much Ado About Nothing 2.3:
    Balthasar.}
\item
  To shun the heaven that leads men to this hell,\footnote{Sonnet 129.
    \textbf{shun} persistently avoid, ignore, or reject (someone or
    something) through antipathy or caution}\\The wild sea of my
  conscience, I did steer.\footnote{Henry VIII 2.4: King.}
\item
  Through the forest I have gone\footnote{Midsummer Night's Dream 2.2:
    Puck.}\\To make some special instance special-blest.\footnote{Sonnet
    52. \textbf{blest} blessed, made holy, consecrated}
\item
  Clouds and eclipses stain both moon and sun---\footnote{Sonnet 35.
    \textbf{eclipse} an obscuring of the light from one celestial body
    by the passage of another between it and the observer or between it
    and its source of illumination \textbf{stain} mark or discolour with
    something that is not easily removed}\\I'll be as patient as a
  gentle stream.\footnote{Two Gentlemen of Verona 2.7: Julia.}
\item
  For I must tell you friendly in your ear:\footnote{As You Like It 3.5:
    Rosalind.}\\The forest walks are wide and spacious.\footnote{Titus
    Andronicus 2.1: Aaron.}
\item
  As plays the sun upon the glassy streams,\footnote{Henry V 1.2:
    Suffolk.}\\Awake the pert and nimble spirit of mirth.\footnote{Midsummer
    Night's Dream 1.1: Theseus. \textbf{pert} lively \textbf{nimble} (of
    the mind) able to think and understand quickly \textbf{mirth}
    cheery, joy}
\item
  Full merrily the humble-bee doth sing,\footnote{Troilus and Cressida
    5.10: Pandarus. \textbf{merrily} in a cheerful way}\\`The more I
  give to thee, the more I have.'\footnote{Romeo and Juliet 2.2: Juliet.}
\item
  The sea all water, yet receives rain still---\footnote{Sonnet 135.}\\God
  be thank'd, there is no need of me.\footnote{Richard III 3.7:
    Gloucester.}
\item
  Out of this nettle, danger, we pluck this flower, safety,\footnote{Henry
    IV, Part I 2.3: Hotspur. \textbf{nettle} a herbaceous plant which
    has jagged leaves covered with stinging hairs \textbf{pluck} take
    hold of (something) and quickly remove it from its place}\\And make
  us heirs of all eternity.\footnote{Love's Labour Lost 1.1: King.
    \textbf{heirs} those who inherit \textbf{eternity} infinite or
    unending time; timelessness}
\item
  That's a valiant flea that dare eat his breakfast on the lip of a
  lion.\footnote{Henry V 3.7: Orleans. \textbf{valiant} possessing or
    showing courage or determination}\\Whilst I am bound to wonder, I am
  bound to pity too.\footnote{Cymbeline 1.6: Iachimo. \textbf{whilst}
    while \textbf{bound} restricted or confined to a specified place or
    thing}
\item
  Full many a glorious morning have I seen,\footnote{Sonnet 33.
    \textbf{glorious} having a striking beauty or splendour}\\For I
  impair not beauty being mute.\footnote{Sonnet 83. \textbf{impair}
    weaken or damage (something, especially a faculty or function)
    \textbf{mute} refraining from speech or temporarily speechless}
\item
  A little fire is quickly trodden out;\footnote{Henry VI, Part III 4.8:
    Clarence. \textbf{trodden} having set one's foot down on top of}\\All
  losses are restored, and sorrows end.\footnote{Sonnet 30.
    \textbf{restored} given back}
\item
  My friends were poor, but honest; so's my love.\footnote{All's Well
    That Ends Well 1.3: Helena.}\\In life's uncertain voyage, I will
  some kindness do them.\footnote{Timon of Athens 5.1: Timon.}
\item
  Let's take the instant by the forward top\footnote{All's Well That
    Ends Well 5.3: King. \textbf{take the instant by the forward top}
    tug occasion by the forelock (hair at the front of the head),
    i.e.~make good use of the present moment.}\\And do whate'er thou
  wilt swift-footed Time.\footnote{Sonnet 19.}
\item
  One minute, nay, one quiet breath of rest.\footnote{King John 3.4:
    Pandulph.}\\A kingdom for it was too small a bound.\footnote{Henry
    IV, Part I 5.4: Prince Henry. \textbf{bound} a territorial limit; a
    boundary}
\item
  With sun and moon, with earth and sea's rich gems,\footnote{Sonnet 21.}\\Buy
  terms divine in selling hours of dross.\footnote{Sonnet 146.
    \textbf{terms divine} touch no-birth no-death/the
    ultimate/favourable terms from God \textbf{hours of dross} time
    wasted devoted to material pleasures}
\item
  Sweet are the uses of adversity\footnote{As You Like It 2.1: Duke
    Senior. \textbf{uses} the value or advantages of something}\\Over
  whose acres walk'd those blessed feet.\footnote{Henry IV, Part I 1.1:
    King Henry IV. \textbf{acres} a unit of land area equal to 4,840
    square yards (0.405 hectare)}
\item
  Men of great worth resorted to this forest\footnote{As You Like It
    5.4: Jaques de Boys. \textbf{worth} high value or merit
    \textbf{resorted} turn to and adopt (a course of action, especially
    an extreme or undesirable one) so as to resolve a difficult
    situation}\\As many fresh streams meet in one salt sea.\footnote{Henry
    V 1.2: Canterbury.}
\end{enumerate}

\vspace{1.4cm}

\begin{figure}[htbp]
\centering
\includegraphics[width=15.5cm]{images/frontispiece.png}
\end{figure}

\end{document}
